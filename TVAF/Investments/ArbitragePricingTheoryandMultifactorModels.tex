\documentclass{article}
\usepackage{graphicx} % Required for inserting images
\usepackage{booktabs} % Para mejor calidad en las tablas
\usepackage{amsmath}  % Para soporte matemático si se necesita
\usepackage{multirow} % Para usar \multirow en tablas
\usepackage[utf8]{inputenc}
\usepackage[T1]{fontenc}
\usepackage[spanish]{babel}
\usepackage[margin=2.5cm]{geometry}
\usepackage{amsmath}
\usepackage{amssymb}
\usepackage{pgfplots}

\setlength{\parindent}{0pt}

\title{
    \textbf{Teoria de Valuacion de Activos Financieros} \\ 
    Resumen y notas de la clase
    \\
    \small Bibliografia: Investments - Bodie 13th edition
}
\author{Ezequiel Telias}
\date{}

\begin{document}

\maketitle

\section{Arbitrage Pricing Theory and Multifactor Models of Risk and Return - Capitulo 10}
El model de factores descompone los retornos en sistematicos e idiosincraticos. Existe un problema de tomar
al riesgo sistematico con un unico factor cuando no lo es. Se menciona el riesgo sistematico como la fuentes de
 la prima de riesgo. Existen otras fuentes de riesgo como tasas de interes, inflacion, etc. Entonces, otra representacion de 
 riesgo sistematico puede ser un buen refinamiento del model de unico factor y nos permite tener un mayor entendimiento de los retornos.

\end{document}