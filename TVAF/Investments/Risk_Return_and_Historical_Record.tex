\documentclass{article}
\usepackage{graphicx} % Required for inserting images
\usepackage{booktabs} % Para mejor calidad en las tablas
\usepackage{amsmath}  % Para soporte matemático si se necesita
\usepackage{multirow} % Para usar \multirow en tablas
\usepackage[utf8]{inputenc}
\usepackage[T1]{fontenc}
\usepackage[spanish]{babel}
\usepackage[margin=2.5cm]{geometry}
\usepackage{amsmath}
\usepackage{amssymb}
\usepackage{pgfplots}

\setlength{\parindent}{0pt}

\title{
    \textbf{Teoria de Valuacion de Activos Financieros} \\ 
    Resumen y notas de la clase
    \\
    \small Bibliografia: Investments - Bodie 13th edition
}
\author{Ezequiel Telias}
\date{}

\begin{document}

\maketitle


\section{Risk, Return, and the Historical Record - Capitulo 5}
Se expresa el Effective Annual Rate  (EAR), una tasa que tiene en cuenta la capitalizacion de los intereses (o el interes compuesto)
 usando la formula de retornos lineales, siendo P(T) el precio pagado con un maturity date T.
\[
r(T) = \frac{P(T)}{100} - 1
\]

EAR explicitamente se usa para intereses compuestos. En contraste, se introduce el Annual Percentage Rate (APR)
, ignorando este interes compuesto. 
\[
1 + EAR = \left(1 + \frac{APR}{n}\right)^{n}
\]
\[
APR = n \times \left[(1 + EAR)^{\frac{1}{n}} - 1\right]
\]
\\
\textbf{Continuous Compunding o Compuesto continuo}
A medida que los intereses aumentan, la pregunta es como van a diverger estos rates. La formula EAR se acerca a un 
compuesto continuo.
\[
\ln(1 + EAR) = r_{cc}
\]

\subsection{Risk and Risk Premium - Riesgo y prima de riesgo}
Holding period return o HPR, se define como: 
\[
HPR = \frac{\text{Ending Price} - \text{Beginning Price} + \text{Cash Dividend}}{\text{Beginning Price}}
\]

El retorno porcentual de los dividendos se llama \textbf{dividend yield}.
\\

\textbf{Expected Return and Standard Deviation}
Para calcular la distribucion probabilistica, primero calculamos la media de retorno E(r).
\[
E(r) = \sum_{s} p(s) r(s)
\]

La varianza entonces, es:
\[
\mathrm{Var}(r) = \sigma^{2} = \sum_{s} p(s) \left[r(s) - E(r)\right]^{2}
\]


Es importante destacar que el desvio estandar no distingue entre buenas o malas sorpresas. Trata a todas las desviaciones
desde la media. Igualmente cuanto mas simetrica es a la media, el desvio, es mas razonable medir el riesgo.
\\

\textbf{Excess Returns and Risk Premiums - Retorno residual y prima de riesgo}
\\
Se mide la recompensa como la diferencia entre el ESPERADO HPR y la tasa libre de riesgo, que es la tasa presente en
assets como fondos de money market o bonos. Esta diferencia es la prima de riesgo, por ejemplo, la tasa libre de riesgo
 es de 4\% y el retorno esperado de otro asset es de 9\%, entonces, la prima de riesgo es de 5\%.  El retorno residual, entonces, es el retorno
 real recibido (Concepto ex post). 
\\

\textbf{The Reward-to-Volatility (Sharpe) Ratio}
\\
Se presume que un inversor esta mas interesado en los retornos residuales esperados que pueden ganar pasando de un bono a un portfolio mas 
arriesgado.
\[
\text{Sharpe Ratio} = \frac{\text{Risk Premium}}{\sigma_{\text{Excess Return}}}
\]
\[
S = \frac{E(r) - r_f}{\sigma(r - r_f)}
\]
\\

\subsection{Deviations from Normality and Tail Risk}
La normalidad de los retornos residuales simplifica en exceso la seleccion de Porftolios. Que pasa si los grandes retornos negativos son mas 
probables que los grandes positivos? La medida estandar de esta asimetria es llamada Skew. Asi como la varianza, es el valor cuadrado
del promedio del desvio, el skew es el cubo de la desviacion promedio. Una desviacion negativa sigue asi una ves hecho el calculo. Entonces,
podemos ver extemos malos resultados.

\[
\text{Skew} = \frac{\text{Average}\left[(R - \bar{R})^{3}\right]}{\sigma^{3}}
\]
\begin{center}
\begin{tikzpicture}
\begin{axis}[
    domain=-4:8,
    samples=200,
    axis lines=middle,
    xlabel={$x$}, ylabel={$f(x)$},
    legend style={at={(0.5,-0.15)},anchor=north},
    width=12cm,
    height=7cm
]
% Distribución simétrica (Normal)
\addplot[blue, thick] {exp(-x^2/2)/sqrt(2*pi)};
\addlegendentry{Simétrica (skew = 0)}

% Sesgo positivo
\addplot[green, dashed] {0.4*exp(-((x-1)^2)/3)};
\addlegendentry{Skew positivo}

% Sesgo negativo
\addplot[red!70!black, dotted] {0.5*exp(-((x+1)^2)/1.5)};
\addlegendentry{Skew negativo}
\end{axis}
\end{tikzpicture}
\end{center}

\newpage

Otra importante separacion de la normalidad es la \textbf{Kurtosis}, que tiene en cuenta la probabilidad de los valores extremos en cualquiera de los lados.
 Alta Kurtosis, significa que hay mas probabilidad en las colas de la distribucion que la que se predice en la distribucion normal.
\\

\[
\text{Kurtosis} = \frac{\text{Average}\left[(R - \bar{R})^{4}\right]}{\sigma^{4}} - 3
\]

\begin{center}
\begin{tikzpicture}
\begin{axis}[
    domain=-5:5,
    samples=200,
    axis lines=middle,
    xlabel={$x$}, ylabel={$f(x)$},
    legend style={at={(0.5,-0.15)},anchor=north},
    width=12cm,
    height=7cm
]
% Distribución normal (kurtosis = 3)
\addplot[blue, thick] {exp(-x^2/2)/sqrt(2*pi)};
\addlegendentry{Normal (kurtosis = 3)}

% Alta kurtosis (leptocúrtica)
\addplot[red, dashed] {1.4*exp(-x^2/1.2)/sqrt(1.2*2*pi)};
\addlegendentry{Alta kurtosis}

% Baja kurtosis (platicúrtica)
\addplot[green!70!black, dotted] {0.7*exp(-x^2/3)/sqrt(3*2*pi)};
\addlegendentry{Baja kurtosis}
\legend{El grafico no esta bien}
\end{axis}
\end{tikzpicture}
\end{center}

\textbf{Value at Risk}
El VaR es una medida para calcular las perdidas extremas en casos dentro del 1 a 5 porciento de la distribucion normal. Este valor
 puede ser de gran utilidad cuando los retornos no son una distribucion normal.

\textbf{Conclusiones}
Long term: A largo plazo, los retornos se asemejan mas a lo que es una distribucion lognormal, que significa que el logaritmos
de el valor final del portfolio es normalmente distribuida. Entonces, se sigue el enfoque de usar los retornos compuestos continuos.

Formulas:

\[
\bar{r}_{\text{arit}} = \frac{r_1 + r_2 + \cdots + r_n}{n}
\]

\[
\bar{r}_{\text{geom}} = \left[(1 + r_1)(1 + r_2) \cdots (1 + r_n)\right]^{\frac{1}{n}} - 1
\]

\[
r_{cc} = \ln(1 + \text{Effective Annual Rate})
\]

\[
E(r) = \sum_{i} p_i r_i
\]

\[
\mathrm{Var}(r) = \sum_{i} p_i \left(r_i - E(r)\right)^2
\]

\[
\sigma = \sqrt{\mathrm{Var}(r)}
\]

\[
S = \frac{E(r_P) - r_f}{\sigma_P}
\]

\[
r_{\text{real}} = \frac{1 + r_{\text{nominal}}}{1 + \text{inflation}} - 1
\]

\[
r_{\text{real (cc)}} = r_{\text{nominal}} - \text{inflation}
\]

\end{document}