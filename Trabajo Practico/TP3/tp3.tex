\documentclass{article}
\usepackage{graphicx} % Required for inserting images
\usepackage{booktabs} % Para mejor calidad en las tablas
\usepackage{amsmath}  % Para soporte matemático si se necesita
\usepackage{multirow} % Para usar \multirow en tablas
\usepackage[utf8]{inputenc}
\usepackage[T1]{fontenc}
\usepackage[spanish]{babel}
\usepackage[margin=2.5cm]{geometry}
\usepackage{amsmath}
\usepackage{amssymb}
\setlength{\parindent}{0pt}

\title{
    \textbf{Teoria de Valuacion de Activos Financieros} 
}
\author{Ezequiel Telias}
\date{}

\begin{document}

\maketitle

\section{Ejercicio 1}
Se supone que un activo es sensible a los movimientos de uno o más factores de mercado M. También, se asume que los eventos inesperados, específicos de la empresa, están incorrelacionados.
 La única fuente de covarianza entre cualquier par de securities es su dependencia común al retorno de mercado.


\section{Ejercicio 2}
 Teniendo en cuenta, $\beta_1$ = 0.85, $\beta_2$ = 1.30m $\sigma_f^2$ = 1.30, se utiliza la siguiente formula para calcular la covarianza:
 \[
\begin{aligned}
\mathrm{Cov}(r_i, r_j) &= \mathrm{Cov}(\beta_i m + \varepsilon_i, \beta_j m + \varepsilon_j) = \beta_i \beta_j \sigma_m^2
\end{aligned}
\]
\[
\begin{aligned}
\beta_i \beta_j \sigma_m^2 = 0,09945
\end{aligned}
\]

\section{Ejercicio 3}

Dada la siguiente informacion, se construira un porfolio con ponderaciones de $2/3$ para el activo X y $1/3$ para Y.
\begin{table}[h!]
\centering
\begin{tabular}{lcc}
\hline
\textbf{Activo} & \textbf{Varianza Residual} & \textbf{Cov}(e\textsubscript{x}, e\textsubscript{y}) \\
\hline
Activo X & 0.02 & 0.01 \\
Activo Y & 0.06 & -- \\
\hline
\end{tabular}
\end{table}

Asumiendo los supuestos de LFM donde la covarianza de los activos tiene que ser igual a 0, se utilizara la siguiente formula:
\[
\sigma^2(\varepsilon_p) = \sum_{j=1}^{N} w_j^2 \, \sigma^2(\varepsilon_j) 
\]
\[
\sigma^2(\varepsilon_p) = (\frac{3}{2})^2 * 0.02 + (\frac{1}{2})^2 * 0.06 = 1,56E-02
\]

Para el siguiente punto se toma la covarianza como 0.01, Multiplicando la matriz de covarianza con las ponderaciones, teniendo como resultado 2,00E-02
. 

\[
\begin{pmatrix}
0.02 & 0.01 \\
0.01 & 0.06
\end{pmatrix}
\]

\section{Ejercicio 4}
Teniendo en cuenta la siugiente estimacion.
\[
r_j = 0.03 + 1.3 r_M + (\varepsilon_j) 
\]

a) Siendo $r_i$ el rendimiento esperado, si el factor de mercado cae 2\%, entonces $r_i$:

\[
r_i = 0.03\% + 1.3 * -2\% = 0,4\%
\]

b) La ecuacion estimada es una funcion lineal asi que un buen grafico para representarla puede ser uno de dispersion mostrando la regresion lineal calculada.
\\

c) Teniendo en cuenta la estimacion inicial, se da lo siguiente:

\[
0,35\% = 0.03 + 1.3 * -0.5\% + \varepsilon_j
\]
\[
0,35\% = 0.03 - 0,0065 + \varepsilon_j 
\]
\[
0,35\% = 0.0365 + \varepsilon_j
\]
\[
 \varepsilon_j = -3\%
\]

Se nota una respuesta negativa de la empresa respecto al mercado.

\section{Ejercicio 5}

Para el siguiente ejercicio se toma en cuenta la siguiente informacion:

\begin{table}[h!]
\centering
\begin{tabular}{lccc}
\toprule
\textbf{Activo} & \(\beta\) & \(\sigma^2_r\) & \(\sigma^2_{\varepsilon}\) \\
\midrule
A & 0.5 & 0.04 & 0.0625 \\
B & 1.5 & 0.08 & 0.2825 \\
\bottomrule
\end{tabular}
\end{table}

a) Para calcular la sensibilidad del portfolio utilizo la siguiente formula:

\[
\beta_p = \sum_{i=1}^{N} w_i \beta_i
\]

\[
\beta_p = 0.5*0.5 + 0.5*1.5 = 1
\]

b) Cumpliendo los supuestos, con una covarianza de los activos igual a 0, entonces, para calcular la varianza total, comienzo con
la varianza idiosincratica del porfolio.

\[
\sigma^2_{\varepsilon_p} = \sum_{i=1}^{N} w_i^2 \, \sigma^2_{\varepsilon_i}
\]

\[
\sigma^2_{\varepsilon_p} = 0.5^2*0.0625 + 0.5^2*0.2825 = 0.03
\]


Para calcular la varianza de mercado, $\sigma_M^2$, despejo la ecuacion de $\sigma^2_{r_i}$ .


\[
\sigma^2_{r_B} = \beta_B^2 \sigma_M^2 + \sigma^2_{\varepsilon_B}
\]

\[
\sigma_M^2 = \frac{\sigma^2_{r_B} - \sigma^2_{\varepsilon_B} } { \beta_B^2}
\]

\[
\sigma_M^2 = 0.09
\]

Entonces, dada la siguiente ecuacion, podemos calcular la varianza total del portfolio:
\[
\sigma_p^2 = \beta_p^2 \sigma_M^2 + \sigma^2_{\varepsilon_p}
\]

\[
\sigma_p^2 = 1 * 0,09 + 0,03 = 0,12
\]


\section{Ejercicio 6}

Teniendo los datos de la columna de volatilidad y Correlacion con PM, se van a calcular las siguientes columnas
 con $\sigma_M^2$ igual a 0,0016.
\\

\begin{tabular}{lcccccc}
\hline
\textbf{Activo} & \textbf{Volatilidad $\sigma$} & \textbf{Correlación con PM} & \(\mathbf{b}\) & \textbf{Riesgo Sistematico} & \textbf{Riesgo Id.} \\
\hline
Blue Axe     & 0,006 & 0,9 & 0,135 & 0,00292\%& 0,0007\% \\
Black Rock   & 0,006 & 0,3 & 0.045 & 0,00032\% & 0,0033\%\\
Silver Light & 0,006 & 0,0 & 0 & 0 & 0,0036\%\\
\hline
\end{tabular}
\\

Para calcular los betas, utilice la siguiente funcion:
\[
\beta_i = \rho_{i,M} \times \frac{\sigma_i}{\sigma_M}.
\]

Para calcular el riesgo sistemico:

\[
= \beta_i \sigma_M^2
\]

Para el riesgo idiosincratico:


\[
\sigma^2_{\varepsilon_i} = \sigma^2_i - \beta_i\sigma^2_M
\]



\section{Ejercicio 7}

Para el siguiente ejercicio se toma en cuenta la siguiente informacion, una correlacion de ABC y XYZ de 0,5. La volatilidad del P es menor a 10\%:

\begin{table}[h!]
\centering
\begin{tabular}{lcc}
\toprule
\textbf{Activos} & \(\rho_{ABC,PM}\) & \(\sigma^2_r\)  \\
\midrule
ABC & 0.0 & 10\% \\
XYZ & 0.5 & 20\% \\
\bottomrule
\end{tabular}
\end{table}


a) Utilizo la siguiente formula:
\[
\beta_i = \rho_{i,M} \times \frac{\sigma_i}{\sigma_M}.
\]


Para el caso del activo ABC, al tener una correlacion nula nos da lo siguiente:

\[
\beta_{ABC} =0 * \frac{0,1}{\sigma_M} = 0.
\]

Para el caso del activo XYZ, tenemos la siguiente ecuacion con $\sigma_M$ menor a  10\%. Entonces para estimar, 
tomamos inicialmente $\sigma_M$ como 10\%:

\[
\beta_{ABC} = 0,5 * \frac{0,2}{0,1} = 1.
\]

Como resultado, $\beta_{ABC}$ es mayor a 1.


b) La covarianza de dos activos la podemos calcular con la siguiente formula al tener los riesgos idiosincraticos 
incorrelacionados.


\[
 \mathrm{Cov}(r_{ABC}, r_{XYZ}) = \beta_{ABC} \times \beta_{XYZ} \times \sigma_M^2
\]

Con $\beta_{XYZ}$ mayor a 1, $\sigma_M^2$ mayor a 10\%, pero siendo $\beta_{ABC}$ igual a 0, como resultado obtenemos:

\[
 \mathrm{Cov}(r_{ABC}, r_{XYZ}) = 0
\]

c) Para calcular la covarianza real, vamos a tener en cuenta la formula de calculo de la correlacion.

\[
\mathrm{Cov}(r_{ABC}, r_{XYZ}) = \beta_{ABC} \times \beta_{XYZ} \times \sigma_M^2
\]


\[
 \mathrm{Corr}(r_{ABC}, r_{XYZ}) = \frac{\beta_{ABC} \times \beta_{XYZ} \times \sigma_M^2}{\sigma_{ABC} \times \sigma_{XYZ}}
\]
\\
Entonces, tenemos como resultado la siguiente formula:

\[
 \rho_{ABC,XYZ}= \frac{\mathrm{Cov}(r_{ABC}, r_{XYZ})}{\sigma_{ABC} \times \sigma_{XYZ}}
\]


\[
\mathrm{Cov}(r_{ABC}, r_{XYZ}) = \rho_{ABC,XYZ} \cdot \sigma_{ABC} \cdot \sigma_{XYZ}
\]

\[
\mathrm{Cov}(r_{ABC}, r_{XYZ}) = 0.5 \times 0.10 \times 0.20 = 0.010
\]


d) Se calcula el beta del portfolio con ponderaciones 40\% en ABC y 60\% en XYZ. Tomando en cuenta, los betas calculados en 
el punto b y teniendo en cuenta que $\beta_{XYZ}$ es > 1. 

\[
\beta_p = \sum_{i=1}^{N} w_i \beta_i
\]
\[
\beta_p = 0.4 \times 0 + 0.6 \times 1 > 0.6 
\]
Tenemos como resultado el beta del porfolio mayor a 0.6. 
\\

e) Para resolver el siguiente ejercicio utilizamos la formula de markowitz con una suma de la varianza multiplicada por su peso.
\[
\sigma_P^2 = w_{ABC}^2 \sigma_{ABC}^2 + w_{XYZ}^2 \sigma_{XYZ}^2 + 2 w_{ABC} w_{XYZ}  \mathrm{Cov}(ABC, XYZ)
\]
\[
\sigma_P^2 = 0,4^2 \times 0,1^2 + 0,6^2 \times0,2^2 + 2 \times 0,4 \times 0,6  \times 0,010
\]
\[
\sigma_P^2 = 0,0208
\]
\\

f) Dada la siguiente ecuacion, tomamos la covarianza como 0 teniendo en cuenta las caracteristicas de un LFM:

\[
\sigma_P^2 = w_{ABC}^2 \sigma_{ABC}^2 + w_{XYZ}^2 \sigma_{XYZ}^2 
\]
\[
\sigma_P^2 = 0,016
\]


\section{Ejercicio 8}
Un LFM es un modelo que descompone los retornos en sistematicos (con unico o multiple factor) e idiosincraticos, teniendo en cuenta la sensibilidad de un activo. A diferencia de Markowitz, necesita un set mas pequeñe de muestra. Existe un problema de tomar
al riesgo sistematico con un unico factor cuando no lo es. Existen otras fuentes de riesgo como tasas de interes, inflacion, etc. 
Entonces, otra representacion de riesgo sistematico puede ser un buen refinamiento del modelo de unico factor y nos permite tener un mayor entendimiento de los retornos.


\section{Ejercicio 9}
Se tiene la siguiente funcion que estima el rendimiento con el modelo LFM de 2 factores sobre el activo H.
Los factores son, rendimiento del portfolio de mercado y la sospresa del crecimien

\[
r_H = 0.5 + 0.8 r_M + 0.2 (g - g^e) + \varepsilon_H
\]

Con $r_M$ = 0,05 y $(g - g^e)$ = 0,02, tenemos el resultado de $r_H$ = 0,544, estimando un retorno residual de 0\%.
\\

b) Con mercado en baja de 2\% porciento sin sorpresa:
\[
r_H = 0.5 + 0.8 \times 0.2 = 0,48
\]


\section{Ejercicio 10}
Tomando la covarianza como 0, me queda la siguiente funcion de varianza de portfolio teniendo en cuenta tambien la idiosincratica.
\[
\sigma^2_p = \sum_{i=1}^{N} w_i ( \beta_{1i}^2 \sigma_{F_1}^2 + \beta_{2i}^2 \sigma_{F_2}^2 )   
\]


\[
\sigma^2_{\varepsilon_p} = \sum_{i=1}^{N} w_i^2 \sigma^2_{\varepsilon_i}
\]

c) Teniendo un set grande de datos y muy diversificado, la varianza idiosincratica, tiende a 0 y puede ser omitida, ya que el numero no impacta realmente al resultado del
rendimiento del porfolio, siendo esta una variable independiente del activo. 

\section{Ejercicio 11}
Teniendo en cuenta la siguiente informacion, se calculara la beta del activo 1.

\[
\beta_2 = 1,2
\]
\[
\sigma^2(r_m)= 0,3162
\]
\[
\mathrm{Cov}(r_{1}, r_{2}) = 0,09
\]


\[
\beta_1 = \frac{\mathrm{Cov}(r_1, r_2)}{\beta_2 \, \sigma^2(r_M)} = \frac{0.09}{1.2 \times 0.3162} \approx 0.237
\]

\section{Ejercicio 12}
Dada la siguiente informacion, asumiendo que la varianza del factor M es 0,06:

\[
\begin{array}{lcccc}
\hline
\textbf{Activo} & w_i & \beta & \text{Retorno esperado} &  \sigma^2_r \\
\hline
\text{Blue Axe} & 0.25 & 0.50 & 0.40 & 7\% \\
\text{Black Rock} & 0.25 & 0.50 & 0.25 & 5\% \\
\text{Silver Light} & 0.50 & 1.00 & 0.21 & 7\% \\
\hline
\end{array}
\]

a) Para calcular la varianaz residual, utilizo la siguiente formula para cada activo:
\[
\sigma_i^2 = \beta_i^2 \sigma_m^2 + \sigma^2(\varepsilon_i)
\]
\[
\sigma^2(\varepsilon_i) = \sigma_i^2 - \beta_i^2 \sigma_m^2  
\]

El resultado, es el siguiente:
\[
\begin{array}{lccccc}
\hline
\textbf{Activo} & w_i & \beta & \text{Retorno esperado} &  \sigma^2_r & \sigma^2(\varepsilon_i) \\
\hline
\text{Blue Axe} & 0.25 & 0.50 & 0.40 & 7\%  & 4\% \\
\text{Black Rock} & 0.25 & 0.50 & 0.25 & 5\% & 2\% \\
\text{Silver Light} & 0.50 & 1.00 & 0.21 & 7\%  &  1\% \\
\hline
\end{array}
\]

b) Se calcula el beta, el rendimiento esperado y la varianza del portfolio.
\[
\beta_p = \sum_{i=1}^{N} w_i \beta_i
\]
\[
\beta_p = 0.25 \times 0,5 + 0.25 \times 0,5 + 0.5 \times 1 = 0,75
\]
\\

\[
\sigma_{r_P}^2 = (\sum_{j=1}^{n} w_j \beta_j)^2 \sigma_{r_M}^2 + \sum_{j=1}^{n} w_j^2 \sigma_{\varepsilon_j}^2
\]
\[
\sigma_{r_P}^2 = 3,69\%
\]

\[
E(r_P) = \sum_{i=1}^{N} w_i \, E(r_i) = 27\%
\]

c) Siguiendo el enfoque de Markowitz la varianza seria la siguiente, para completar la 
tabla se tomaron las varianzas del primer punto:
\[
\begin{array}{lccc}
 & \text{Blue Axe} & \text{Black Rock} & \text{Silver Light} \\
\hline
\text{Blue Axe} & \sigma_{\text{BA}}^2 & 0.020 & 0.035 \\
\text{Black Rock} & 0.020 & \sigma_{\text{BR}}^2 & 0.035 \\
\text{Silver Light} & 0.035 & 0.035 & \sigma_{\text{SL}}^2 \\
\end{array}
\]

\[
\sigma_P^2 = \sum_i \sum_j w_i w_j \mathrm{Cov}(r_i, r_j) = 0,045
\]

d)
¿Qué conjeturas pueden argumentarse si se halla que la covarianza real difiere
de los supuestos del LFM?
Se puede tener como conjetura que el LFM modelado esta mal especificado teniendo factores 
que estan correlacionados entre si. Tambien, puede ser posible que existan factores adicionales relevantes que no se consideran.

\end{document}